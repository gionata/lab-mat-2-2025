% -*- mode: latex; coding: utf8; TeX-master: giochi-di-numeri-massi.tex -*-
% !TeX root = giochi-di-numeri-massi.tex
% !TeX encoding = UTF-8 Unicode
% !Tex TS-program = LuaLaTeX
\documentclass[withtimes]{easychair}
\usepackage[utf8]{inputenc}
\usepackage[T1]{fontenc}
\usepackage[italian, english]{babel}
\usepackage[autostyle,italian=guillemets]{csquotes}
\usepackage[backend=biber,style=numeric,hyperref]{biblatex}
\usepackage{longtable}
\usepackage{booktabs}
\usepackage{ellipsis}
\usepackage[mode=buildnew]{standalone}% requires -shell-escape
\usepackage{tikz}
\usepackage{pygmentex}
\addbibresource{giochi-di-numeri-massi.bib}
\usepackage{subcaption}
% \usepackage{graphicx}
\usepackage{caption}
\usepackage{doc}

%% Front Matter
\pagestyle{empty}
\fontfamily{garamond}

\title{Giochi di Numeri:\\Pensiero Computazionale e Aritmetica}

\author{
Gionata Massi%\inst{1}\thanks{Ha svolto il ruolo di formatore esperto.}
}

% Institutes for affiliations are also joined by \and,
\institute{
   Istituto di Istruzione Superiore Savoia Benincasa\\
   Ancona, Italia\\
   \email{gionata.massi@savoiabenincasa.it}
}

\authorrunning{Massi}

\titlerunning{Giochi di numeri}

\begin{document}

\maketitle

\selectlanguage{italian}

\begin{abstract}
  Questo articolo presenta un'esperienza didattica focalizzata sullo sviluppo del pensiero ricorsivo attraverso la costruzione formale dell'aritmetica partendo da tre procedure primitive: il \emph{predicato zero}, che applicato ad un argomento restituisce \emph{vero} se l'argomento è 0, e le funzioni \emph{successore} e \emph{predecessore} che applicate ad un argomento $n$ restituiscono, rispettivamente, $n+1$ e $n-1$. Il corso guida gli studenti nella definizione ricorsiva delle operazioni aritmetiche fondamentali (addizione, sottrazione, moltiplicazione e divisione intera) e degli operatori di confronto (\emph{minore o uguale}, \emph{maggiore o uguale}, \emph{uguale}, \textellipsis). L'azione didattica, ispirata dal testo ``The Little Schemer'' e concretizzata con l'uso del linguaggio \emph{Scheme}, mira a evidenziare la profonda connessione tra ragionamento logico-deduttivo, problem-solving informatico e procedure effettive di calcolo. Vengono discussi gli obiettivi, la metodologia adottata, gli strumenti utilizzati e le riflessioni sull'efficacia di tale percorso nel promuovere una comprensione più profonda dei concetti informatici e stimolare l'interesse verso le discipline scientifico-tecnologiche.
\end{abstract}

\section{Introduzione}\label{introduzione}

Nel panorama educativo odierno, la necessità di sviluppare il \emph{pensiero computazionale} è ampiamente riconosciuta come una competenza chiave per affrontare le sfide del XXI secolo. Tale pensiero non si limita alla mera capacità di programmare, ma abbraccia una serie di abilità cognitive essenziali, tra cui la scomposizione di problemi, il riconoscimento di pattern, l'astrazione e, in modo preminente, il \emph{pensiero ricorsivo}. Quest'ultimo, in particolare, costituisce la spina dorsale di numerosi algoritmi e strutture dati in informatica, ma la sua comprensione profonda spesso rappresenta una sfida per gli studenti.

Il corso nasce proprio con l'intento di affrontare questa sfida, proponendo un percorso didattico che, anziché introdurre la ricorsione in contesti complessi, la esplora a partire dalle sue radici più fondamentali: la costruzione dell'aritmetica stessa. L'idea centrale è che i ragionamenti sottostanti alle operazioni numeriche, se formalizzati in modo ricorsivo, rivelano una struttura universale applicabile alla risoluzione di un'ampia gamma di problemi computazionali. Questo approccio non solo rafforza le basi matematiche, ma getta anche le fondamenta per una comprensione meno superficiale e più consapevole degli oggetti di studio e dei metodi dell'informatica.

Il corso è stato sviluppato nell'ambito del progetto scolastico dell'IIS Savoia Benincasa ``Citizen scientists of the future'' (C.U.P. H34D23002330006), finanziato dal Piano Nazionale di Ripresa e Resilienza (PNRR), Missione 4 -- Istruzione e Ricerca, Investimento 3.1 ``Nuove competenze e nuovi linguaggi'' ed è stato rivolto agli studenti della classe seconda del Liceo Matematico.

Le scelte metodologiche sono fortemente influenzate dalle opinioni personali dell'autore sulla necessità di usare un linguaggio di programmazione per esprimere le idee e di ricondurre il pensiero astratto alla manipolazione di oggetti fisici, quando possibile, per migliorare la comprensione degli oggetti di studio e dei metodi informatici.

\begin{quote}
    "Il calcolatore era (ed è ancora) un nuovo e meraviglioso concetto filosofico e matematico. Il calcolatore è ancora più rivoluzionario come idea che come congegno pratico che modifica la società -- e tutti sappiamo quanto abbia cambiato la nostra vita. Perché lo dico? Perché il calcolatore cambia l'epistemologia, modifica il significato del verbo «comprendere». A mio giudizio, si capisce qualcosa solo se si è capaci -- noi, non altri! -- di scriverne il programma. Altrimenti non si ha una vera comprensione, si crede soltanto di capire."
    \small{-- Gregory Chaitin, \textit{Alla ricerca di Omega}, Adelphi Edizioni. 2007}
\end{quote}

\section{Obiettivi}\label{obiettivi-del-corso}

Gli obiettivi generali del corso sono stati definiti per andare oltre la mera acquisizione di nozioni, mirando a un impatto più profondo sulla formazione degli studenti:

\begin{itemize}
 \item Comprendere meno superficialmente gli oggetti di studio e i metodi   della disciplina informatica.%
 % L'esposizione alla costruzione formale dell'aritmetica e alla logica ricorsiva sottostante mira a demistificare concetti che altrimenti potrebbero apparire come ``magie'' della tecnologia, rivelandone la rigorosa base matematica e logica.
  \item Stimolare l'interesse verso le materie tecnico scientifiche e, in   particolare, verso l'informatica.
  \item Rendere tangibili e ``calcolabili'' alcuni concetti astratti come quelli alla base dell'aritmetica.
  \item Cogliere la stretta relazione tra pensiero scientifico e informatica. % sviluppo   tecnologico. Attraverso la formalizzazione delle idee e la loro formalizzazione in un linguaggio di programmazioe, gli studenti possono apprezzare come il ragionamento astratto e la logica siano alla base dell'innovazione tecnologica.
 \item Comprendere le strutture fondamentali dei ragionamenti logico-deduttivi.
 \item Padroneggiare il linguaggio logico-formale per risolvere problemi di varia natura.
 \item Esercitare e affinare il ragionamento deduttivo attraverso la ricorsione.
\end{itemize}

% La ricorsione è un potente strumento di ragionamento deduttivo. Il corso fornisce un contesto pratico per esercitare e affinare questa capacità.

\section{Metodologia e Strumenti}\label{metodologia-e-strumenti}

Per perseguire questi obiettivi, è stato scelto un approccio basato sulla formalizzazione delle idee attraverso il linguaggio \emph{Scheme}. Esso ha una sintassi semplice che può essere interpretata con un modello di riscrittura delle espressioni, cosa che rende il linguaggio facile da visualizzare e comprendere. %, e può essere impiegato immediatamente per esplorare concetti come la ricorsione, l'astrazione e la manipolazione simbolica.
% Alcuni importanti corsi di introduzione all'informatica utilizzano Scheme come linguaggio, e si citano due testi importanti nel loro corso di introduzione: ``Structure and Interpretation of Computer Programs''~\cite{} e ``How to Design Programs''~`\cite{}.

Il corso si basa in gran parte sui primi quattro capitoli di ``The Little Schemer''~\cite{Friedman1995}, un testo che introduce il concetto di ricorsione in modo giocoso e incrementale, attraverso una serie di domande e risposte che guidano il lettore a ``scoprire'' le definizioni ricorsive. Questa scelta didattica è fondamentale: non ci sono ricette pronte ma si deve procedere in un processo di scoperta attiva, dove lo schema cognitivo viene costruito e consolidato attraverso la costruzione della definizione che valuta la risposta.

Per facilitare l'apprendimento e ridurre la barriera linguistica, le procedure di Scheme usate nel corso sono state tradotte in italiano \footnote{\texttt{cond} e \texttt{zero?} sono parole chiave che vengono usate nel corso ma non necessitano di traduzione.}, come mostrato nella tabella \ref{tbl:traduzioni}:

\begin{table}
  \centering
  \begin{tabular}{ll} \toprule
  \multicolumn{1}{c}{Scheme} & \multicolumn{1}{c}{Traduzione}\\\midrule
  \texttt{car} & \texttt{primo} \\
  \texttt{cdr} & \texttt{resto} \\
  \texttt{cons} & \texttt{anteponi} \\
  \texttt{eq?} & \texttt{uguale?} \\
  \texttt{list} & \texttt{lista} \\
  \texttt{list?} & \texttt{lista?} \\
  \texttt{null?} & \texttt{lista-vuota?} \\
  \texttt{\textquotesingle{}()} & \texttt{lista-vuota} \\\bottomrule
  \end{tabular}
  \caption{Traduzione semantica di alcune parole chiave}\label{tbl:traduzioni}
\end{table}

Questa traduzione ha permesso di concentrarsi sui concetti logici senza aggiungere un carico cognitivo eccessivo dovuto alla memorizzazione di termini in lingua straniera oppure legati ai registri di un vecchio processore.

Altro elemento tenuto in considerazione durante le lezioni è stato quello di offrire mediatori attivi e iconici per comprendere strutture e processi. Il mediatore attivo, nel caso dell'algebra, è rappresentato dai \emph{regoli Cuisenaire} mentre come mediatori iconici si sono usati vari diagrammi. Questi rappresentano la struttura sintattica delle espressioni simboliche ma anche l'evoluzione del processo (macchina nozionale).

\section{Il Percorso Didattico}\label{il-percorso-didattico}

Il percorso costruisce progressivamente concetti, illustrando e sperimentando attivamente le operazioni fondamentali su atomi e liste di atomi per arrivare alla definizione delle operazioni aritmetiche. Come nel testo, le attività sono introdotte da domande e le risposte sono calcolate formalizzandone la procedura.

\subsection{Atomi, Liste, Espressioni}\label{strutture-dati-e-espressioni}

Il percorso si apre con la rappresentazione degli \emph{atomi} e delle \emph{liste} (fig.~\ref{fig:lista}). Viene illustrato l'isomorfismo con le \emph{espressioni simboliche} e queste vengono riscritte con l'operatore prefisso. L'espressione viene visualizzata come \emph{albero sintattico} (fig.~\ref{fig:albero}). La rappresentazione non è risultata intuitiva e si è introdotta la rappresentazione per cerchi della valutazione \footnote{Si veda il progetto \url{https://www.bootstrapworld.org/}{BootStrap}, corso \emph{Bootstrap: Algebra}, lezione \emph{Order of Operations} dove vengono introdotti i \emph{Circles of Evaluation}.} (fig.~\ref{fig:cerchio}). Viene fornito come macchina nozionale il modello per l'applicazione di procedure di SICP~\cite[par.~\mbox{1.1.5} The Substitution Model for Procedure Application]{Abelson1996}.

\begin{figure}
    \centering

    \begin{subfigure}[t]{0.8\textwidth}
      \centering
      \includestandalone[width=.8\textwidth]{img/lista}
      \caption{Espressioni come liste\label{fig:lista}}
    \end{subfigure}

    \vspace{1em}

    \begin{minipage}{0.8\textwidth}
      \centering
      \begin{subfigure}[b]{0.35\textwidth}
        \centering
        \includestandalone[width=.8\textwidth]{img/albero}
        \caption{Liste come alberi\label{fig:albero}}
      \end{subfigure}
      \quad
      \begin{subfigure}[b]{0.35\textwidth}
        \centering
        \includestandalone[width=.8\textwidth]{img/cerchio}
        \caption{Alberi come cerchi\label{fig:cerchio}}
      \end{subfigure}
    \end{minipage}

  \caption{Mediatori iconici per le espressioni simboliche}
\end{figure}

% \begin{figure}
%   \centering
%   \includestandalone[width=.8\textwidth]{img/albero}
%   \caption{Rappresentazione di liste come alberi}
%   \label{fig:albero-sintattico}
% \end{figure}
%
% \begin{figure}
%   \centering
%   \includestandalone[width=.8\textwidth]{img/cerchio}
%   \caption{Rappresentazione di alberi come cerchi di valutazione}
%   \label{fig:albero-sintattico}
% \end{figure}

% Queste visualizzazioni sono state utilizzate in aula per aiutare gli studenti a ``vedere'' la struttura sottostante delle espressioni e comprendere il processo di valutazione ricorsiva.

\subsection{Predicati, Espressione Condizionale e Liste di Atomi}\label{atomi-operatori-predicati-e-liste-di-atomi}

Il modulo procede per domande da risolvere scrivendo e valutando espressioni in Scheme usando \href{https://www.wescheme.org/}{WeScheme}\footnote{Ambiente Scheme disponibile all'indirizzo: \url{https://www.wescheme.org/}.} con le traduzioni della tab.~\ref{tbl:traduzioni}. Le domande richiedono la manipolazione di atomi, valori booleani ed espressioni simboliche. Viene introdotta l'espressione condizione con \texttt{cond} e con questa, gradualmente, il concetto di ricorsione. Esempi sono predicati come \texttt{lista-di-atomi?} (fig.~\ref{fig:lista-di-atomi?}) e \texttt{membro?}.

\begin{figure}[ht]
  \centering
  \begin{pygmented}[lang=scheme]
; Predicato lista-di-atomi?
(define lista-di-atomi?
  (lambda (l)
    (cond
      [(lista-vuota? l) #t]
      [(atomo? (primo l)) (lista-di-atomi? (resto l))]
      [else #f])))

; Domanda: (+ 1 2) è una lista di atomi?
; (lista-di-atomi? '(+ 1 2))
; -> (lista-di-atomi? '(1 2))
; -> (lista-di-atomi? '(2))
; -> (lista-di-atomi? '())
; -> #t
; Risposta: vero!
  \end{pygmented}
  \caption{Predicato \texttt{lista-di-atomi?}}
  \label{fig:lista-di-atomi?}
\end{figure}

% Esempi di definizioni chiave includono:
%
% \begin{figure}
%   \centering
%   \begin{pygmented}[lang=scheme]
%     ; Predicato atomo?
%   (define atomo?
%     (lambda
%       (x)
%       (and (not (pair? x)) (not (null? x)))))
%
%   ; Primo elemento di una lista
%   (define primo
%     (lambda (l)
%       (cond
%         [(not (list? l)) (display "'primo' è definito solo per le liste")]
%         [(null? l) (display "'primo' non è definito per la lista vuota")]
%         [else (car l)])))
%   \end{pygmented}
%   \caption{Esempi di definizioni chiave}
%   \label{fig:esempi-definizioni-chiave}
% \end{figure}

\subsection{Manipolazione Ricorsiva delle Liste}\label{manipolazione-ricorsiva-delle-liste}

Vengono proposte manipolazioni più complesse delle liste, definendo funzioni come \texttt{rimuovi-un-membro}, \texttt{inserisci-a-destra} e \texttt{sostituisci} (fig.~\ref{fig:sostituisci})
Ogni funzione è stata costruita ricorsivamente, enfatizzando il pattern base della ricorsione: un caso base (solitamente la \texttt{lista-vuota}) e un caso ricorsivo che opera
sul \texttt{primo} elemento e invoca la funzione sul \texttt{resto} della lista. Questi esercizi consolidano la comprensione del \emph{pattern ricorsivo} e della sua applicazione a problemi di manipolazione dati.

\begin{figure}[ht]
  \centering
  \begin{pygmented}[lang=scheme]
; Sostituisci nella lista-di-atomi il vecchio con il nuovo
(define sostituisci
  (lambda (nuovo vecchio lista-di-atomi)
    (cond
     [(lista-vuota? lista-di-atomi) lista-vuota]
     [(uguale? (primo lista-di-atomi) vecchio)
            (anteponi nuovo (resto lista-di-atomi))]
     [else (anteponi
                (primo lista-di-atomi)
                (sostituisci nuovo vecchio (resto lista-di-atomi)))])))

; Domanda: Sostituisci con * il + nella lista di atomi (+ 1 2)
; (sostituisci '* '+ '(+ 1 2))
; -> (anteponi '* '(1 2))
; -> '(* 1 2)
; Risposta: la lista (* 1 2)
  \end{pygmented}
  \caption{Procedura \texttt{sostituisci}}
  \label{fig:sostituisci}
\end{figure}


\subsection{Giochi di Numeri: La Costruzione Ricorsiva dell'Aritmetica}\label{giochi-di-numeri-la-costruzione-ricorsiva-dellaritmetica}

Il culmine del corso è stato la costruzione delle operazioni aritmetiche utilizzando solo le funzioni primitive \texttt{zero?}, \texttt{s} (successore) e \texttt{p} (predecessore). Questo approccio, radicalmente diverso dalla semplice memorizzazione di regole, ha permesso agli studenti di ``ricostruire'' l'aritmetica da zero, apprezzandone la struttura ricorsiva intrinseca.

Le definizioni delle procedure per l'aritmetica sono in fig.~\ref{fig:primitive-aritmetica}.

\begin{figure}
  \centering
  \begin{pygmented}[lang=scheme]
; Successore di un numero naturale n
(define s
  (lambda (n)
    (+ n 1)))

; Predecessore di un numero naturale n
(define p
  (lambda (n)
    (- n 1)))

; Zero? è già definito dal linguaggio Scheme.

  \end{pygmented}
  \caption{Definizione delle primitive per la costruzione ricorsiva dell'aritmetica sugli interi}\label{fig:primitive-aritmetica}
\end{figure}

Attraverso i regoli Cuisenaire vengono mostrate le operazioni di \texttt{addizione}, \texttt{sottrazione}, \texttt{moltiplicazione}, \texttt{divisione} intera (in fig.~\ref{fig:divisione-regoli} si mostra il processo di sottrazioni ripetute per il calcolo di 12 diviso per 4). Gli studenti devono trasformare le regole di manipolazione dei cubi bianchi in procedure simboliche (fig.~\ref{fig:divisione-scheme}).

\begin{figure}[ht]
    \centering
    \includestandalone[width=.8\textwidth]{img/divisione}
    \caption{La divisione di 12 per 4 illustrata con i regoli Cuisenaire\label{fig:divisione-regoli}}
\end{figure}


\begin{figure}
  \centering
  \begin{pygmented}[lang=scheme]
; Minore? n di m
(define minore?
  (lambda (n m)
    (cond
      [(zero? m) #f]
      [(zero? n) #t]
      [else (minore? (p n) (p m))])))

; Sottrazione tra n e m
(define sottrazione
  (lambda (n m)
    (cond
     [(zero? m) n]
     [else (sottrazione (p n) (p m))])))

; Divisione di n per m
(define divisione
  (lambda ( n m )
    (cond
     [(minore? n m ) 0]
     [else (s (divisione (sottrazione n m) m ))])))
; Domanda: Divisione di 12 per 4
; (divisione 12 4)
; -> (s 8 4)
; -> (s (s 4 4))
; -> (s (s (s (0 4))))
; -> (s (s (s 0)))
; -> (s (s 1))
; -> (s 2)
; -> 3
; Risposta: 12 diviso 4 è pari a 3!
  \end{pygmented}
  \caption{Divisione intera}\label{fig:divisione-scheme}
\end{figure}



Queste definizioni, spesso sorprendenti per gli studenti abituati alle operazioni dirette dei linguaggi di programmazione, hanno evidenziato
come concetti complessi possano essere ridotti a un insieme minimo di primitive e regole ricorsive. Il percorso ha incluso anche la definizione di operatori di confronto (\texttt{maggiore?}, \texttt{minore?}, \texttt{stesso-numero?}) e altre funzioni sulle liste e i numeri, come \texttt{lunghezza}.

\section{Conclusioni}\label{conclusioni}

L'esperienza del percorso ha dimostrato l'efficacia di un approccio che pone il pensiero ricorsivo al centro dell'apprendimento dell'informatica. Costruire l'aritmetica ``da zero'' in un contesto formale ma accessibile come quello offerto da Scheme e ``The Little Schemer'' ha permesso agli studenti di:

\begin{itemize}
\item   Sviluppare un'intuizione più profonda per la ricorsione. % Anziché vederla come un costrutto sintattico, l'hanno sperimentata come un principio fondamentale di risoluzione dei problemi, applicabile sia all'aritmetica che alla manipolazione simbolica.
\item   Migliorare le capacità di ragionamento logico-deduttivo. % La necessità  di definire ogni operazione in termini di casi base e ricorsivi ha affinato le loro abilità di pensiero strutturato.
\item   Comprendere la natura formale dell'informatica. % L'uso di un linguaggio come Scheme, con la sua chiara corrispondenza con il lambda calcolo, ha svelato la base matematica rigorosa dietro la programmazione.
\item  Aumentare la consapevolezza sulla relazione tra matematica e informatica. % Il corso ha reso esplicita la continuità tra il pensiero matematico e le tecniche computazionali, superando la percezione di   queste discipline come entità separate.
\end{itemize}

Sebbene la curva di apprendimento iniziale possa essere ripida per alcuni, l'esperienza ha rivelato che, una volta compreso il paradigma ricorsivo, gli studenti acquisiscono una potente lente attraverso cui analizzare e risolvere una vasta gamma di problemi. Questo approccio non solo arricchisce il loro bagaglio di conoscenze tecniche, ma coltiva
anche una mentalità critica e analitica.% , essenziale per navigare nel mondo sempre più complesso dell'informatica.

In conclusione, questo percorso rappresenta un modello promettente per l'insegnamento del pensiero computazionale, dimostrando che un ritorno alle basi formali e la costruzione incrementale dei concetti, guidati da strumenti adeguati, possono portare a una comprensione più profonda e duratura delle discipline informatiche.

\label{sect:bib}
\printbibliography

\end{document}
